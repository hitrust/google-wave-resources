\input sem2-macros
\usepackage{helvet} \renewcommand{\familydefault}{\sfdefault} 
\usepackage[footnotesize]{caption}
\usepackage{wrapfig}

\setcounter{milestone}{-1}

\begin{document}
%%%%%%%%%%%%%%%%%%%%%%%%%%%%%%%%%%%%%
\psetheader{Google Wave Extension Assignment}
%%%%%%%%%%%%%%%%%%%%%%%%%%%%%%%%%%%%%

\medskip


%%%%%%%%%%%%%%%%%%%%
% General overview
%%%%%%%%%%%%%%%%%%%%
\section{General Overview}

According to Google, Wave is "what email would look like if it is
invented today". Also according to Google, there are serious drawbacks
with working on a project using emails~\cite{googlewave}:

\begin{enumerate}
\item \textbf{Propagation of email leads to multiple copies and versions}.  
Different users will be getting different copies of messages overtime
and some may not be updated on the latest development.
\item \textbf{Rich content such as maps, photo slideshows, and video clips
cannot be embedded in the body of an email}. The only way to do it is
through the email's attachment feature.
\item \textbf{Difficult to reply to a particular section of an email}. 
To reply to a certain section of an email, one needs to look for the
section and quote it manually.
\item \textbf{Difficult to address to specific people in a group email}.
To do that, users have to manually delete unwanted recipients from the
mailing list.
\end{enumerate}

In view of these problems, Google Wave is designed as a better tool to
facilitate such collaborative processes. Using Google Wave, the users
will be able to
\begin{enumerate}
\item \textbf{Eliminate the problem of receiving multiple copies 
of outdated messages}. There will only be a single copy of the message hosted online and any changes will be updated real time. There is also
a playback function that allows users to trace how the message has
been modified over time by different users.
\item \textbf{Allows rich content to be embedded within the message}.
\item \textbf{Reply to a section of a message inline}. To reply to a 
specific section of the message, the user can just click on it and
type in his or her reply.
\item \textbf{Reply to specific people within a group}. Similar to 
forums, the users can start private threads within a group message.
\end{enumerate}

Currently, Wave is in an invite-only preview stage. This preview
release was launched in September 2009. In the coming months, it is
anticipated that the number of users will be growing exponentially. If
you can understand the users' needs, you can leverage on this
opportunity as pioneer developers to make a difference!

To better understand Google Wave, you may wish to check out the
following introductory video (about 1 hour
long): \url{http://youtu.be/v_UyVmITiYQ}

\section{Acknowledgement}

We would like to express our gratitude to Google for their help and
support in making this assignment possible, especially {\em Stephanie
Liu}, who helped us obtained the sandbox accounts and {\em Pamela Fox},
who provided us with invaluable feedback on the assignment.

\section{Grading and Admin}

This assignment can be done in groups of three or four students.  If
you are unable find yourself a group, you will be randomly assigned to
one.

We will not be providing step-by-step instructions. Instead, we will
only list the key concepts you need to master (also known as
"milestones"). We will also provide some related tips, references and
a little bit of help to get you started. These milestones constitute
70\% of the assignment's grade.

Like the previous assignment, if you find that some of the proposed
milestones do not make sense for the application you intend to build,
you can petition to replace them with some other deliverables. You are
to explain why we should agree to your petition, and submit your
petition at least one week before the assignment is due. Your petition
is subjected to approval.

While the milestones may be easy to meet, simply meeting them will not
give you the maximum credit. We ask for quality submissions, not
run-of-the-mill work.

To score the coveted remaining 30\%, we would like to see how you can
put your creativity to the test and explore all pores of your
originality tissues. We choose not to restrict your potential by
insisting on any particular sequence. We strongly recommend that you
blow us away with your creativity.

Please do not hesitate to approach the friendly CS3216 staff if you
need further assistance. If you have questions with Google Wave,
please post your questions in the IVLE Discussion Forum or better
create a Wave and discuss it with a Wave! Whatever works. :-)

%%%%%%%%%%%%%%%%%%%%
% Objectives
%%%%%%%%%%%%%%%%%%%%
\section{Objectives}

In this assignment you are required to deploy a Google Wave
application.  Your goal is to demonstrate that you can implement a
Wave gadget, a Wave robot, or both.

While we believe (and Google believes too) that the most sophisticated
extensions will be combinations of gadgets and robots, and we did
consider making every group do both. However, we decided that we did
not want to be prescriptive so as to provide you with maximum
flexibility and minimal restrictions on what you can do. Extra credit
will be award to groups who incorporate both in a seamless way to do
something cool.

Remember, your goal is not to do a lot of work. Your goal is to use
this opportunity to ``make a difference''. If you can make a
difference by just doing a little bit of work, that's fine with us.

Before you begin, do spend some time understanding the requirements
for the assignment. Also, try to get some sleep before you start on
this assignment. To help you avoid losing sleep, we have included a
detailed grading scheme at the end of this handout.

%%%%%%%%%%%%%%%%%%%%
% Stage 0
%%%%%%%%%%%%%%%%%%%%
\section{Introduction}

Welcome! This assignment comprises of 2 weeks of intensive learning
that provides you with another opportunity to express your creativity.

To enrich the users' experience, the Wave platform is open to external
developers. These new features and functionalities that external
developers develop are called {\em Wave extensions}. Like Facebook
applications, Wave users can add these extensions by clicking on an
installation button.

There are two types of Wave extensions~\cite{googlewave_wiki}:
\begin{enumerate}
\item \textbf{Robots}. Robots are automated participants within a wave 
which can talk to users and interact with waves. It can be used to
provide information such as stock quotes from external sources.
\item \textbf{Gadgets}. A gadget is an application where users can join 
to interact with each other. You can think of the gadget as being
similar to iGoogle gadgets or Facebook applications.
\end{enumerate}
To be even more accurate, an extension does not need to involve only a
gadget or robot~\cite{yellow_highlight}. It can also involve both a
robot and a gadget~\cite{connect4}.

You can choose to develop either a robot, a gadget, or both. If you
choose to build a wave gadget, you need to complete seven mandatory
milestones (4-10). If you choose to build a robot, you are expected to
complete three compulsory milestones (11-13). In addition, there are
eight milestones (milestones 0, 1, 2, 3, 14, 15, 16 and 17) that are
common to both paths of development. Out of these eight milestones,
milestone 16 is optional and milestones 2 and 3 are not graded.

Before you start, please take a moment to check out the Extension
Design Principles~\cite{googlewave_extension_principles} (and see
milestone 14).

\alertbox{Reminder}{You are given a choice as to what you want to build
(either a gadget or a robot, or both :-P). To ensure that you will make the
right choice, please take some time to read through the entire
assignment before thinking about what you want to develop. You are
also welcome to do some simple experiments to understand what Wave can
and cannot do.}

%%%%%%%%%%%%%%%%%%%%
% Stage 1
%%%%%%%%%%%%%%%%%%%%
\section{Thinking hard about your Wave Application}
\begin{center}
\begin{quote}
No charge for awesomeness!\\
\hfill -- Kungfu Panda
\end{quote}
\end{center}

Since you are among the first Google Wave extension developers in the
world, you are likely to be excited about sharing what you are doing
with your friends. Mention ``Wave developer'' to your friends, and
they will be impressed.

As exciting as developing on a new platform may seem, it also
has problems associated with being new. Why would you want to develop
on the Wave platform instead of another more established platforms,
such as Facebook?

Building a killer Wave application requires more than just technical
skills. In CS3216, we expect you to think very hard about what you're
trying to do. You should not be building a Wave application just
because you need to submit this piece of homework.

You should choose an application that truly exploits the potential of
the Wave platform. For example, you should have a good answer to the
question of why developing on other platforms is not the best solution
to fulfill your application's objectives.

In CS3216 (and life in general), execution matters. Identifying a good
idea is the first step, deciding on which path of development to take,
that ensures maximum success for your application, is the
next. Deciding whether to build a gadget or a robot (or both) is
crucial in your execution. Thus, we expect you to come up with a good
reason for it.

\milestone{Milestone}{Describe your application and explain how you 
have exploited the Wave platform to achieve your application's
objectives, i.e.\ why is Wave a more suitable platform than other
platforms like Facebook to develop your chosen application?  You can
choose to develop either a gadget and/or a robot. Explain your
choice.}

Unlike the Facebook application you have developed in the first assignment,
your Wave application does not have a ready-made social network to
leverage on. It is no good to have a killer app that nobody
uses. Hence, you will also be expected to think a little harder about
how you plan to ``market'' your app to potential users. You must
identify your target users, determine the relevance of your
application to them (i.e.\ why should they care about your
application) and explain how you plan to reach out and persuade them
to use your application.

In order to promote the use of your application, good marketing
strategies are crucial in raising awareness of your application among
the targeted users. After introducing potential users to your
application, how would you try to persuade them to continue using the
application, and perhaps, even share or introduce it to others?  What
value do the users derive from using your application?

Ideally, you should also think of ways to provide motivation for users
to promote your application to their friends. The application should
be designed in such a way that while individual user may derive some
values using your application, it is in their interest to promote your
application as more users will make your application more valuable to
them.

Your efforts in promoting your app should also take into consideration
that Wave is currently in its infancy and that users get their wave account
only via invitation. Working under this restriction, how would you
ensure that your target users will be attracted to your app?

It is also reasonable to expect that the number of Wave users will
grow rapidly in the foreseeable future. Given this trend, how would
you take advantage of the growth to increase the number of people
using your app?

\milestone{Milestone}{Describe your target users. Given that Wave 
is a new platform, explain how you plan to promote your application to
attract your target users.}

Before you start doing anything with Google Wave, you need an account,
which you should already have at this point.  There are 2 types of
Wave accounts: regular and sandbox accounts. They are almost
identical, except that the sandbox account is meant for
developers. The sandbox account will have earlier access to new
features and APIs, which will only work in the Sandbox environment.  The
Sandbox environment also has a debug interface which is not available
in the regular Wave environment. You should develop an extension in
the Sandbox environment first and move it to the regular Wave
environment only when you're done. If you decide to develop an
application that uses Sandbox-environment-specific APIs, you don't
need to move to the regular Wave environment. Just indicate this
clearly in the submission README file.

% \begin{enumerate}
% 
% \end{enumerate}


\milestone{Milestone}{ Set up your Google Wave and sandbox account. 
Update the profiles with your real name (so that we can identify you)
and upload a picture \textit{(Not graded)}.}

After a discussion with your groupmates, you should have an idea of
what you want to build. Now, it is time to pick a name for your Wave
extension. It is also a good idea to write a short description about
what your extension can do. Just a short paragraph will do (3-5
lines).

\milestone{Milestone}{ Pick a good name for your Google Wave extension, 
along with a description for your Google Wave extension.
\textit{(Not graded)}.}

\section{Wave Gadget}

This section is not relevant if you are building a Wave Robot, but you
are encouraged to read through and understand what a gadget can
do. After all, if you decided to change your mind with regards to the
kind of extension you would like to build, it is always better to
change your mind early.

\subsection{Getting Started}

The following is a simple ``Hello World'' gadget. This should take you
less than 5 minutes to complete.

Create a file name {\tt hello.xml} with the following content:

\begin{verbatim}
<?xml version="1.0" encoding="UTF-8" ?>
<Module>
  <ModulePrefs title="Hello Wave">
    <Require feature="wave" /> 
  </ModulePrefs>
  <Content type="html">
    <![CDATA[     
       Hello, Wave!
    ]]>
  </Content>
</Module>
\end{verbatim}
or download the file
from: \url{http://gadget-doc-examples.googlecode.com/svn/trunk/wave/hello.xml}

Upload the newly-created file to your webserver (this could be any
publicly accessible webserver, i.e.\ your SoC unix hosting works for
this. Each group will be provided with more Amazon credits for this
assignment, so use it wisely.

Login to your Wave account, create a new Wave and click on this
button \includegraphics{images/insertGadget_icon.png} (only available
if you're in edit mode).

Go to the URL you have noted down, you should see the text ``Hello,
Wave!''  in your wave. Congratulations, you have just built your first
gadget :-).

%Explain on this simple Wave gadget
Take a look at this segment of code again: \begin{verbatim}
<![CDATA[     
   Hello, Wave!
]]>
\end{verbatim}
It contains the bulk of the gadget, including your gadget's logic and
presentation. Next, we will run through how you can add these elements
into the gadget.

\subsection{Building your gadget}

The {\em Hello World} gadget only has 1 line of text and is quite
boring. In this section, you will learn how to use CSS and JavaScript
to make your gadget look better.

\begin{verbatim}
<![CDATA[
  Your code goes here
]]>

\end{verbatim} 

Any code inside {\tt <![CDATA[]]> } is treated as the content of the
body tag in HTML file. It can contain CSS and JavaScript, or refer to
them.

\subsubsection{Presentation}

Your team (or maybe just your user interface designer) should spend
some time designing a good UI. A good UI is the key factor that
attracts users to your gadget. Although the functionality of your
gadget is important, the way that it provides the functionality is
just as important. A gadget that is difficult to use won't be used.
Period.  It won't matter how technically superior your gadget is or
what functionality it provides. If your users don't like it, they
simply won't use it. Seriously, do spend some time getting it
right. In most cases, you'll know immediately if your UI makes or
breaks it. It's common sense(!).

According to web standards, you should use CSS to style your
webpage. Since the Wave gadget is just another webpage embedded in a
Wave conversation, the same principles applies here.  CSS stands for
Cascading Style Sheets, a language to determine the formatting and
layout of a web page. It provides a way to separate content from
presentation, thus helping us to organize the code in a much more
efficient manner.

All your styling should be contained within a CSS file (you can link
an external file to your gadget) or clearly defined at the beginning
of your gadget. It is very bad practice to mix CSS into HTML code.

\milestone{Milestone}{ Style different UI components within the gadget
using CSS in a structured way (i.e.\ marks will be deducted if you
submit messy code). Explain why your UI design is the best possible UI
for your application.}

After you have created a fancy UI, you need to teach your gadget what
to do.

\subsubsection{Behaviour}

Wave gadgets use JavaScript to define its logic. Besides pure
JavaScript functions, it has access to 2 sets of APIs:

\begin{itemize}
\item Google Gadget API: \url{http://code.google.com/apis/gadgets/docs/reference/}
\item Wave Gadget API: \url{http://code.google.com/apis/wave/extensions/gadgets/reference.html}
\end{itemize}

Google Gadget API allows you to build gadgets that can run on multiple
sites, such as iGoogle, Orkut (Google's social networking platform),
Google Wave or your own site. You can go
to \url{http://www.google.com/ig} and start adding stuff to see what
Google Gadget is about.

A gadget on iGoogle will most likely run fine on Google Wave, except
that it would not be able to take advantage of many new features
offered by Wave (such as real time synchronization and multi-user
environment). By using Wave Gadget API, you can make use of these
features to for your gadget to interact with the Wave
conversation/participants.

Wave gadget works with callback functions. A callback function is a
function that is passed to another function (in the form of a pointer
to the callback function) so that the second function can call
it. This is simply of way of making the second function more flexible
without the second function needing to know a lot of stuff. By passing
different callback functions, you can get different behaviours. A
function can take multiple callback functions as a parameter and
execute accordingly.

Here are some interesting functions that take another function pointer
as parameters:

\textbf{Google Gadget API}

{\tt gadgets.util.registerOnLoadHandler}: call the callback function
right after the gadget gets loaded.

\textbf{Wave Gadget API}

{\tt setParticipantCallback}: call the callback function every time
the Wave's participant list changes. This is ideal for gadgets that
need to track the participants in a Wave conversation.

{\tt setStateCallback}: call the callback function every time the
Wave's state (mentioned later) changes.

To start, we need to define which function to call when the gadget
is first loaded.  From there, we will continue adding more callback
function.

Let's try to pop up a message when we load the gadget. Copy this chunk
of code and put it on your server:

\begin{verbatim}
<?xml version="1.0" encoding="UTF-8" ?>
<Module>
  <ModulePrefs title="Hello Wave" width="400" height="120">
    <Require feature="wave" /> 
  </ModulePrefs>
  <Content type="html">
    <![CDATA[     

    <script>
      function init() {
        alert('This is a pop up');
      }
      gadgets.util.registerOnLoadHandler(init);
    </script>
    ]]>
  </Content>
</Module>
\end{verbatim}

{\tt gadgets.util.registerOnLoadHandler} defines that the function
init will be called when you first load the gadget. The init function
simply pops up a message in your browser.

By default, wave sets your gadget height at about 1 text-line tall. The width, on the other had, is dynamic as you resize your window. That means if you
don't specify the height in your gadget (see code above), your gadget
will display only the first line.  Wave omits the scrollbar too, so
users will not be able to recognize that there's more information
below. It is recommended that you set the width and height of your
gadget correctly to avoid this situation.

If your gadget requires dynamic height, you could use the dynamic
height feature in Google gadget API. More details are available here:
\url{http://code.google.com/apis/gadgets/docs/ui.html#Dyn_Height}

\subsubsection{Store your data}

Google Wave gadgets typically do not use a relational database to
store the shared state information. Instead, key-value state
is used. In a key-value state, there is no row or column. Every value
is associated with a key name. The only way to access a value is to
know its name.

Before you store any information in the gadget's state, note that the
state is only available within the gadget in a Wave conversation
(e.g.\ a gadget cannot access state of another gadget in the
same/different Wave conversation). Furthermore, the state object is
{\em shared} among every participants within a Wave conversation. Any
participant can make any change to the state any time he or she
wants. You should remember the following when you develop your gadget:

\begin{itemize}
\item Only expose information that is meaningful to everyone on the wave.
\item Do not expose any private information of a user. If you need to,
let them know before they enter such information. Users' privacy is
important.  Violation of this rule will lose your users' trust in no
time.
\item There could be instances where many users try to modify a state 
at the same time.
\end{itemize}

The state object is accessible via Google Wave gadget API. Some basic
APIs that might be of interest: {\em submitValue} (create/update one
key-value pair), {\em submitDelta} (create/update the state delta),
{\em get} (retrieve a value).

\subsubsection{Callback functions}

As mentioned, Wave gadgets work by registering callback
functions. Changes to a wave's state is an important event that your
application may need to response to. Using the API, it is possible to
call a callback function every time the wave's state ({\em
setStateCallback}) or the wave's participants ({\em
setParticipantCallback}) changes. Once a callback function is called,
it means your state is updated or participants list is changed. Your
code inside the callback function will receive the latest change of
your state and participant list. Therefore it is a good idea to put
programming logic inside callback functions.

\milestone{Milestone}{ Show 2-5 gadget API and 
wave gadget API calls in your Wave gadget that you consider to be the
most interesting. Explain how these API are used to do interesting
things in your gadget.}

\milestone{Milestone}{ Store some information in your gadget and
modify/delete them (in an appropriate way).}

{\em \textbf{Note:} if you're stuck with the wave's state, Google has
a nice guide to follow
here} \url{http://code.google.com/apis/wave/extensions/gadgets/guide.html#state}

It is a good idea to modify the state object when your UI component
fires an event. Almost any interaction on any UI component fires up an
JavaScript event. For example:
\begin{itemize}
\item On submitting a form.
\item On clicking a button.
\item On mouse down on a component
\end{itemize}

{\em \textbf{Note:} You cannot submit a form in a wave conversation
since your gadget is running within the wave environment only. You
should bind an event handler to the submit button or the form itself
to run your code on form submission. This would be identical to using a
button and binding an event handler on clicking that button.}

You will most likely want to capture user input via the UI. In
most cases, it makes sense to update the state object after the user
inputs something.  Please do not add an event listener to your
code just because we ask you to do so.  Do remember that every single
event that happens in your gadget contributes towards the user experience.

\milestone{Milestone}{ Cite 2 or 3 examples where you use JavaScript 
to respond to events on UI components. Explain.}

Wave comes with a set of useful built-in features, such as the
realtime updating of state object and spell checking. This means that
you can save yourself time from having to synchronize information
among participants or implement a spell checking mechanism. While
Google has solved some of the hardest problems with online
collaboration, how you make use of these features is something that
you will have to figure out. You should keep these features in mind
when you're building your gadget and use them creatively. How much
creativity is enough?  We'll leave that to your imagination =).

\milestone{Milestone}{ Explain how your gadget makes use of built-in 
wave-features (e.g.\ real time update, playback, etc.).}

%user experience part
\subsubsection{User experience}

Another important part of your gadget is user experience. Please note
that user experience (usually addressed as UX or UE) is different from
user interface (UI). A good UI does not guarantee a good UX at
all. Sometimes, a cool-looking UI can be a disaster because of poor
UX.

User experience encompasses all aspects of the end-user's interaction
with the gadget. The first requirement for a good user experience is
that the gadget allows the users to do what he wants with minimal
fuss. Next, comes simplicity and elegance which will make the gadget a
joy to use.  User experience goes far beyond giving user what they say
they want, or providing a checklist of features. In order to achieve
high-quality user experience in a gadget, there must be seamless
integration of the gadget's functionality, interaction design and
interface design.

User experience is not just the job of the UI designer. Just like a
good UI, you will know if a gadget's UX makes or breaks. It is again,
just common sense. Any team member can contribute to your UX design by
using it and provide feedback and suggestions. Ask your friends to use
it as well to gather more feedback and ideas.

\milestone{Milestone}{ Describe 1-3 user interactions within the gadget 
and explain why those interactions help make the gadget better.}

\subsubsection{Google Analytics}

Just like the Facebook application, you might be interested in the
usage statistics of your gadget. While Facebook applications have
access to Facebook Insight, which provides a lot more information
about your application, the only mechanism you have in gadget is to
embed Google Analytics (or other tracking mechanism) in your
gadget.

However, do note that Google does not recommend the use of Google
Analytics in gadgets, because Google Analytics will record every access
a user makes to a wave conversation.  Google's privacy policy for Wave
states that a user should not reveal that he is on a wave until he is
actually editing the wave. Google is working on Wave to make sure it
complies with this privacy policy.

If you want to include Google Analytics in production gadgets, make
sure that you inform the users about the consequence of adding the
gadget (i.e.\ being tracked even if they are not editing a wave).

\milestone{Milestone}{ Embed Google Analytics on all your pages and give us
a screenshot of the report. (Optional)}

% \subsubsection{Facebook Connect}
% 
% In May 2008, Facebook announced the launch of its latest development
% of Facebook Platform - Facebook Connect\footnote{Facebook Connect.
% \url{http://developers.facebook.com/news.php?blog=1&story=108} }.
% It enables you to implement and offer features of Facebook Platform at
% your website. You can now extract data from the Facebook database to
% enhance the user experience of the users visiting your
% application. For instance, when a user visit your cool website and
% want to discuss it with his friends, he can log into Facebook using
% Facebook Connect instead of logging in via the Facebook main
% site. This offers the advantage of keeping the user in your website
% and leveraging on the wide social network offers by
% Facebook.\footnote{ For more information, please refer
% to: \url{http://developers.facebook.com/connect.php}}
% 
% \milestone{Milestone}{Make use of Facebook Connect to implement
% innovative features. Explain to us how Facebook Connect help you to
% make your gadget better \textit{(Optional)}.}


\section{Wave Robot}

Like the Wave Gadget section, you are not required to read this
section if you intend to build a Wave Gadget. However, we would
recommend that you read it anyway. :-)

{\bf Definition}. {\em A robot is an automated participant on a
wave. A robot can read the contents of a wave, participates in it, modify the wave's contents, add or remove participants
and create new blips and new waves. In short, a robot can perform many
of the actions that any other participant can
perform\footnote{\url{http://code.google.com/apis/wave/extensions/robots/\#Introduction}}.}

\subsection{Getting started}

Currently Google Wave only supports robots hosted on Google App
Engine\footnote{\url{http://code.google.com/appengine/}}.  First
you'll need to register for a Google App Engine account first.

After getting a Google App Engine account, you can create and deploy
up to 10 applications with your account. You cannot change application
name after you have created it, so pick the name wisely. You can see the
list of your applications here:
\url{https://appengine.google.com/}

You can use either Java or Python to develop your robot on Google App
Engine.

%%%%
%Reserve to write Google App Engine tutorial
%%%%

%Robot API
\subsection{Wave Robot API Library}

\begin{itemize}
\item For Java:
\begin{itemize} 
\item Robot library: \url{http://code.google.com/p/wave-robot-java-client/downloads/list}
\item Robot API reference: \url{http://wave-robot-java-client.googlecode.com/svn/trunk/doc/index.html}
\end{itemize}
\item For Python: 
\begin{itemize} 
\item Robot library: \url{http://code.google.com/p/wave-robot-python-client/downloads/list}
\item Robot API reference: \url{http://wave-robot-python-client.googlecode.com/svn/trunk/pydocs/index.html}
\end{itemize}
\end{itemize}

You'll need Java 6 development kit or Python 2.5 or higher installed
on your system. You can check your Java/Python version using the
following commands:

\begin{itemize}
\item Java:  {\tt java -version}
\item Python:  {\tt python --version}
\end{itemize}

If you don't have the correct version, download either JDK 6 or Python
2.6.4 using the following links:

\url{http://java.sun.com/javase/downloads/index.jsp}

\url{http://www.python.org/download/}

\subsection{Robot Identity}

To register for a new application on Google App Engine, go
to \url{https://appengine.google.com/} and follow the instructions.

After registering for your application, you'll be given an application
id application-id.appspot.com. Your robot's address will be
application-id@appspot.com.  This is similar to your wave address
username@googlewave.com. To add your robot into a wave is just like
adding a new human participant.

\subsection{Hello world!}
%python part
\subsubsection{Python}

After you have downloaded your Python API Library, extract it and
rename the folder to {\em waveapi}. Create another folder and move the
{\em waveapi} folder to the newly created one.

We will start building our first robot. This robot will respond when
you add it into a Wave conversation, or when it sees the participants
list changes.

Create a file name {\em app.yaml} in your source directory (which
means it is on the same level with {\em waveapit} folder) with the
following content:

\begin{verbatim}
  application: applicationname
  version: 1
  runtime: python
  api_version: 1

  handlers:
  - url: /_wave/.*
    script: applicationname.py
  - url: /assets
    static_dir: assets
\end{verbatim}

\textbf{Warning: Indentation is important in YAML, you need to make sure the indentation
in your {\em app.yaml} file is correct.}

YAML stand for YAML Ain't Markup Language. It's a human-friendly
data-serialization standard for all programming
languages \footnote{From \url{http://yaml.org/}}.  There are a few
things you might want to take note in this simple example:

\begin{itemize}
\item {\em application}: the name of your application. This must be in 
lowercase and exactly like your appspot ID (Google App Engine use this
application name to recognize your app).
  
\item {\em version}: define the version of your robot. Google App Engine 
allows you to have at most 10 versions of the same app store on Google
App Engine. Login to your Google App Engine, there's a menu call
"Version" in Administration area. You will be able to change the
default version for your app as well as delete some outdated versions.

\item {\em handlers}: specify URL pattern to serve static resources for application.
\end{itemize}

You can see this YAML file as a configuration file in your
application. It tells the Wave client some information about your
robot in order to work correctly.

Now create another file name {\em applicationname.py} (or anything
depends on what you have in your {\em app.yaml}. This is the main code
to respond to different events in a Wave conversation) in your source
directory with the following content:

\begin{verbatim}
  from waveapi import events
  from waveapi import robot
  from waveapi import appengine_robot_runner

  def OnParticipantsChanged(event, wavelet):
    """Invoked when any participants have been added/removed."""
    new_participants = event.participants_added
    for new_participant in new_participants:
      wavelet.reply('\nHi :' + new_participant)

  def OnRobotAdded(event, wavelet):
    """Invoked when the robot has been added."""
    wavelet.reply('\nI'm alive!')

  if __name__ == '__main__':
    myRobot = robot.Robot('appName', 
        image_url='http://appName.appspot.com/icon.png',
        profile_url='http://appName.appspot.com/')
    myRobot.register_handler(events.WaveletParticipantsChanged, OnParticipantsChanged)
    myRobot.register_handler(events.WaveletSelfAdded, OnRobotAdded)
    appengine_robot_runner.run(myRobot)
\end{verbatim}

This is a bit more complicated. The first 3 lines include the Python
robot library into your robot:

\begin{verbatim}
from waveapi import events
from waveapi import robot
from waveapi import appengine_robot_runner
\end{verbatim}

The last part of the code is the main function of your robot. It is
put at the bottom of the file so that it can access the define
functions at the top.

\begin{verbatim}
\end{verbatim}

{\em RegisterHandler} function register a list of callback function to
call when an event happens. A callback function is a function that is
passed to another function (in the form of a pointer to the callback
function) so that the second function can call it. This is a simple
way of making the second function more flexible without the second
function needing to know a lot of stuff. By passing different callback
functions, you can get different behaviour.

In the example above, we register 2 events with their associated
callback functions:
\begin{itemize}
\item {\em WaveletParticipantsChanged}: trigger when list of participants of
a Wave conversation changes.
\item {\em WaveletSelfAdded}: trigger when the robot is added into a Wave 
conversation.
\end{itemize}

The full list of events can be found
here:
\url{http://wave-robot-python-client.googlecode.com/svn/trunk/pydocs/index.html#module-events}

\begin{verbatim}
  myRobot = robot.Robot('appName', 
      image_url='http://appName.appspot.com/icon.png',
      profile_url='http://appName.appspot.com/')
\end{verbatim}

This is the constructor for the robot. You can specify the name of
your robot, profile image, and profile link.

When the matching event occurs, the Wave client will automatically make a call 
to your application to trigger the event in your code.

The only way your robot can work is via these events. Your team will
need to spend some time deciding which event to support and what to do
after the event is fired. In CS3216, we expect you to make use of
these events to make your robot useful. You shouldn't add more event
listeners into your source code just because we ask for it.

\milestone{Milestone}{ Design your robot to respond to 3-4 events. 
Explain in detail how these events serve your robot's purpose.}

One nice thing about robots is that it can access all the information
in a Wave conversation (except for private messages, of course). This
makes robot a perfect candidate to do automated tasks. One common
task you can start with is to watch the participants' message, find
a matching pattern, and respond accordingly - like an emoticon finding robot.

Since Google Wave is an online collaboration tool, sometimes people
just throw a lot of stuff into a Wave conversation. Imagine building a gallery in Google Wave with 100
contributors. Everyone contributes 100 images, which makes a total of
10000 pictures. For a human being to collect all these images, it would
take a lot of time just to save them to his or her harddisk. This is where
Wave robots comes in. A robot can scan through a Wave conversation and
extract all images out a lot quicker than a human being. Furthermore a
robot can be reused.  So if there is an instance where you want to make a
second gallery, it would be a lot more easier.

\milestone{Milestone}{ Extract data from Wave conversation, 
and explain how you will use the data.}

Aside from all the event listeners, you should make use of other APIs
to make your robot more useful. Again, you should only use the APIs
that are really needed.  We will not give credit for poorly-used API
calls that do not contribute to a good system design.

Here is a list of existing Wave robot that you can take a look at and
find your inspiration:
\begin{itemize}
\item \url{http://wave-samples-gallery.appspot.com/results?api=Robots}
\end{itemize}
There are also source codes that you can learn from (One of the
fastest way to learn is by reading other people's code).

\milestone{Milestone}{ Make use of 3-5 API functions in your robot. 
Explain in details how those API functions help to make your robot
more useful.}

\subsubsection{Deploy to Google App Engine}

Download Google App Engine SDK for Python
here: \url{http://googleappengine.googlecode.com/files/google_appengine_1.3.0.zip}

Extract the file to your computer. Make sure you have the correct
version of Python on your computer. Fire up terminal/command line on
your system and enter this command:

\begin{verbatim}
  appcfg.py update [path to your source folder]
\end{verbatim}

You will be asked for your login credential. If there's any error, the
script will output accordingly.  Otherwise it will say that your
application is ready for serving.

There is GUI version of the SDK for Mac and Windows (sorry Linux
 folks) if you
 prefer: \url{http://code.google.com/appengine/downloads.html}

%Java is bad for you! At least as far as wave is concerned%
\subsubsection{And Java?}

We chose to introduce the robots API using Python,
because it generally requires less setup than Java, and is easier to explain
here.

But, you are still welcome to develop with the robots API using Java.
You will need to download the zip files here:

\url{http://code.google.com/p/wave-robot-java-client/downloads/list}

in addition to installing the Eclipse IDE and Google's plugin for it.

Google has a good step by step tutorial here that we recommend you
follow \textbf{closely} to get your first Java robot up:

\url{http://code.google.com/apis/wave/extensions/robots/java-tutorial.html}

The milestones are the same should you decide to use Java. We
reiterate here that in terms of grading, we are not concerned about
which language you use, so pick the language that maximizes your
productivity.

%common for both
\section{Common for Wave Gadget and Wave Robot}

\subsection{Extension Design Principles}

You were asked to consider the Extension Design Principles in
designing your extension~\cite{googlewave_extension_principles}.

\milestone{Milestone}{Explain how your extension complies with these 
design principle. Or explain how and it didn't (as it's not always
easy to do so).}

\subsection{Google Wave Extension Installers}

An extension installer is a xml file that defines a quick way to:

\begin{itemize}
\item Add a gadget to a Wave conversation.
\item Add a participant to a Wave conversation (perfect for adding robot, isn't?).
\item Tag the current wave conversation with an annotation.
\item Create a new wave
\end{itemize}

In the regular Wave account, follow this tutorial to enable extension
installer in your
account: \url{http://code.google.com/apis/wave/extensions/installers/index.html#Production}

The extension installer syntax is fairly simple and we expect you to
figure it out by yourself after reading the tutorial
here: \url{http://code.google.com/apis/wave/extensions/installers/index.html}

An extension installer will be the main way to spread your Wave gadget
and Wave robot, since most people will not remember your gadget URL or
your robot identity. By adding an extension installer, a user can
quickly access your gadget/robot within a few clicks and in a
graphical way.

\milestone{Milestone}{ Create an installer for your gadget/robot. 
It should add your gadget/robot into a Wave conversation
automatically.}

\subsection{Embedding API}
%explain on Embedding API

We have discussed how to build an extension to interact with users in a
Wave conversation. Aside from using wave on the Wave website, it's
possible to use Wave on your own website. Users with a Wave account
can interact with the Wave conversation on your website. Users with the
appropriate permissions can view, reply, edit directly to the Wave on
your website and it gets updated in real time to another user (be it
on googlewave.com or through another site that the same Wave
conversation is embedded in).

The embedded wave has the same characteristics as the conversation on
the wave client. You can do a few things to it - include gadgets, add
a robot to it, and even modify some UI elements to seamlessly
integrate with the look and feel of your website.
%Hello world err
To embed a Wave conversation in your website/blog, you first need to know 
the Wave id. Each Wave is associated with a globally unique id. 

You can use the following method to find a Wave conversation id:
\begin{enumerate}
\item Open the Wave conversation in your browser
\item Copy the URL. It will look like this: \url{https://wave.google.com/wave/#restored:wave:googlewave.com!w%252BWcTtzYqSA}
\item Replace the \%252B part with +
\item Your Wave conversation ID is googlewave.com!w+WcTtzYqSA
\end{enumerate}

After you have got the Wave conversation id, open your web page in a
text editor and fill this in the {\tt <head>} part.

\begin{verbatim}
  <script src='http://wave-api.appspot.com/public/embed.js' 
  type='text/javascript'></script>
  <script>
  function initialize() {
    var wave = new WavePanel('https://wave.google.com/wave/');
    wave.loadWave('YOUR-WAVE-ID');
    wave.init(document.getElementById('placeholder'));
  }
  </script>
\end{verbatim}

The first line includes the wave embed API to your web page, then
 the \emph{initialize} function will load the Wave conversation into
 the element with the id \emph{placeholder}. Please note
 that \emph{placeholder} is an element on your web page, not the wave
 id.

The \emph{initialize} should only run after the Wave Embed API is
fully loaded, otherwise the Wave conversation will not load correctly
or behave weirdly.

To make sure that the page is fully loaded, put an \emph{onload} parameter in the 
body tag:

\begin{verbatim}
  <body onload="initialize();">
\end{verbatim}

Now open your web page again, you should see your Wave
conversation. By default, it uses standard Wave UI. You can modify
some elements to match the theme on your website using the API.

The full API reference can be found
here: \url{http://code.google.com/apis/wave/embed/reference.html}

Google's tutorial on embedding a wave conversation can be found
here: \url{http://code.google.com/apis/wave/embed/guide.html}
%styling UI

\milestone{Milestone}{Make use of Wave Embed API to implement innovative 
features. Explain to us how Wave Embed API helps you to achieve your
project objective. Do note that you need to create an online website
(Optional).}

% \subsection{Other API}

%end part

Congratulations! You are now among the first group of people in the
world to have successfully developed a Wave extension. Since you are first,
we (the Teaching staff) need your help!

Being a new platform, we expect you to suffer a fair share of mishaps
and frustrations. :p You should have discovered that developing in a
new platform is not all that fun after all. There should be glitches
that have yet to be ironed out.  Maybe you also have an opinion on how
the development experience can be improved.

\milestone{Milestone}{What are the shortcomings that you have discovered 
about the Wave platform? How do you think the Wave platform can be
improved?}

You might consider publishing your extension to allow people to use
it. Google has an example gallery, which you can submit your extension
to. To submit your extension to the sample gallery, follow the
instruction
here: \url{https://wave-samples-gallery.appspot.com/submit}. Once you
have a production-quality extension, can you submit it
here: \url{https://wave-extensions-gallery.appspot.com/submit}.

\section{Detailed Grading Scheme}

The grading of the assignment is divided into two components:
satisfying the compulsory milestones(70\%) and coolness factor
(30\%). 

There are three types of compulsory milstones in this assignment:
\begin{enumerate}
    \item Milestones common to both paths of development
    \item Milestones for the wave gadget section
    \item Milestones for the wave robot section
\end{enumerate} 

\subsection{Milestones common to both paths of development}

Regardless of the path of development, these milestones are
mandatory. This section contributes 25\% to the total assignment
grade.

There are eight milestones (milestones 0, 1, 2, 3, 14, 15, 16 and 17)
in this section. Of these eight milestones, milestone 16 is optional
and milestones 2 and 3 are not graded.

Milestones 0, 1, 14, 15 and 17 each contributes 5\% to the total
assignment grade.

\subsection{Milestones for the wave gadget section}

This section contributes 45\% to the total assignment grade.

There are seven milestones (milestone 4-10) in this section.

Milestones 6 and 9 each contributes 7.5\% to the total grade.  The
rest of the milestones each contributes 6\% to the total grade.

\subsection{Milestones  for the wave robot section}
This section contributes 45\% to the total assignment grade.

There are three milestones (milestone 11-13) in this section.  Each
milestone contributes 15\% to the total grade.

If you choose to build both the gadget and the robot, we will be
grading both sections and taking the higher grade of the two sections
as the final grade. The synergy between the interaction of these two
extensions will be taken into account in the coolness factor.

This assignment is due on \textbf{27th Feb 2010}. You
are expected to deliver the following:
\begin{enumerate}
    \item Milestone 1: A page or two of write-up on how the Wave extension helps
    to achieve the objectives of the assignment

    \item Milestone 2: A page or two of description on your target
    users and your marketing strategies for your application

    \item Completion of all remaining compulsory milestones and write-up explaining how we can see that the milestones were achieved.

    \item A short write-up pitching your application to
    the teaching staff, i.e. convince us that your application is so
    good that it deserves all 30\% of the coolness factor
    points. Restriction: no longer than 2 pages (A4), 12pt. font,
    Georgia or equivalent.
\end{enumerate}

\section{Mode of Submission}

The following is the list of deliverables:

\begin{enumerate}
    \item Source code: compress all the source files into a single
    archive (zip/rar/tarball), maintaining the directory structure of
    the source files.

    \item Provide us the URL to your installer, i.e. your gadget/robot must be 
    accessible online somewhere. If you built a gadget, you're required to provide 
    the URL of the gadget (just in case your installer doesn't work). Similarly,
    if you build a robot, you're required to provide the robot identity. If you
    do both, submit both and write a short description on the sequence of adding
    gadget/robot (e.g. your app only works correctly if the gadget/robot is added
    first).
    \item If you gadget/robot only works in sandbox environment, please state it 
    clearly or we might not be able to run it. By default we test your gadget/robot
    in regular Wave environment.
    \item A short readme file containing the list of group members,
    including matriculation numbers, name and a description of the
    contributions of each member to the assignment. Make sure that
    your application name is clearly written in the README file. The
    README file should also contain all the necessary write-ups for
    the milestones for this assignment.

\end{enumerate}

Archive all deliverables into a single archive (zip/rar/tarball) and
name it assignment2-[GroupNo.].{ zip/rar/tarball}. 

As a reminder, if you choose to build a wave gadget, you need to complete seven mandatory milestones. If you choose to build a robot, you are expected to complete three compulsory milestones. In addition, there are six milestones (excluding milestone 15) common to both paths of development.

Good luck and have fun!

\bibliographystyle{abbrv} 
\bibliography{wave} 

\end{document}
